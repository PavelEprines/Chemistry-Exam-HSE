%
%Не забыть:
%--------------------------------------
%Вставить колонтитулы, поменять название на титульнике



%--------------------------------------

\documentclass[a4paper, 12pt]{article} 

%--------------------------------------
%Russian-specific packages
%--------------------------------------
%\usepackage[warn]{mathtext}
\usepackage[T2A]{fontenc}
\usepackage[utf8]{inputenc}
\usepackage[english,russian]{babel}
\usepackage[intlimits]{amsmath}
\usepackage{esint}
%--------------------------------------
%Hyphenation rules
%--------------------------------------
\usepackage{hyphenat}
\hyphenation{ма-те-ма-ти-ка вос-ста-нав-ли-вать}
%--------------------------------------
%Packages
%--------------------------------------
\usepackage{amsmath}
\usepackage{amssymb}
\usepackage{amsfonts}
\usepackage{amsthm}
\usepackage{latexsym}
\usepackage{mathtools}
\usepackage{etoolbox}%Булевые операторы
\usepackage{extsizes}%Выставление произвольного шрифта в \documentclass
\usepackage{geometry}%Разметка листа
\usepackage{indentfirst}
\usepackage{wrapfig}%Создание обтекаемых текстом объектов
\usepackage{fancyhdr}%Создание колонтитулов
\usepackage{setspace}%Настройка интерлиньяжа
\usepackage{lastpage}%Вывод номера последней страницы в документе, \lastpage
\usepackage{soul}%Изменение параметров начертания
\usepackage{hyperref}%Две строчки с настройкой гиперссылок внутри получаеммого
\usepackage[usenames,dvipsnames,svgnames,table,rgb]{xcolor}% pdf-документа
\usepackage{multicol}%Позволяет писать текст в несколько колонок
\usepackage{cite}%Работа с библиографией
\usepackage{subfigure}% Человеческая вставка нескольких картинок
\usepackage{tikz}%Рисование рисунков
\usepackage{float}% Возможность ставить H в положениях картинки
% Для картинок Моти
\usepackage{misccorr}
\usepackage{lscape}
\usepackage{cmap}

% Для Х И М И И

\usepackage{mhchem}



\usepackage{graphicx,xcolor}
\graphicspath{{Pictures/}}
\DeclareGraphicsExtensions{.pdf,.png,.jpg}

%----------------------------------------
%Список окружений
%----------------------------------------
\newenvironment {theor}[2]
{\smallskip \par \textbf{#1.} \textit{#2}  \par $\blacktriangleleft$}
{\flushright{$\blacktriangleright$} \medskip \par} %лемма/теорема с доказательством
\newenvironment {proofn}
{\par $\blacktriangleleft$}
{$\blacktriangleright$ \par} %доказательство
%----------------------------------------
%Список команд
%----------------------------------------
\newcommand{\grad}
{\mathop{\mathrm{grad}}\nolimits\,} %градиент

\newcommand{\diver}
{\mathop{\mathrm{div}}\nolimits\,} %дивергенция

\newcommand{\rot}
{\ensuremath{\mathrm{rot}}\,}

\newcommand{\Def}[1]
{\underline{\textbf{#1}}} %определение

\newcommand{\RN}[1]
{\MakeUppercase{\romannumeral #1}} %римские цифры

\newcommand {\theornp}[2]
{\textbf{#1.} \textit{ #2} \par} %Написание леммы/теоремы без доказательства

\newcommand{\qrq}
{\ensuremath{\quad \Rightarrow \quad}} %Человеческий знак следствия

\newcommand{\qlrq}
{\ensuremath{\quad \Leftrightarrow \quad}} %Человеческий знак равносильности

\renewcommand{\phi}{\varphi} %Нормальный знак фи

\newcommand{\me}
{\ensuremath{\mathbb{E}}}

\newcommand{\md}
{\ensuremath{\mathbb{D}}}



%\renewcommand{\vec}{\overline}




%----------------------------------------
%Разметка листа
%----------------------------------------
\geometry{top = 3cm}
\geometry{bottom = 2cm}
\geometry{left = 1.5cm}
\geometry{right = 1.5cm}
%----------------------------------------
%Колонтитулы
%----------------------------------------
\pagestyle{fancy}%Создание колонтитулов
\fancyhead{}
%\fancyfoot{}
%----------------------------------------
%Интерлиньяж (расстояния между строчками)
%----------------------------------------
%\onehalfspacing -- интерлиньяж 1.5
%\doublespacing -- интерлиньяж 2
%----------------------------------------
%Настройка гиперссылок
%----------------------------------------
\hypersetup{				% Гиперссылки
	unicode=true,           % русские буквы в раздела PDF
	pdftitle={Заголовок},   % Заголовок
	pdfauthor={Автор},      % Автор
	pdfsubject={Тема},      % Тема
	pdfcreator={Создатель}, % Создатель
	pdfproducer={Производитель}, % Производитель
	pdfkeywords={keyword1} {key2} {key3}, % Ключевые слова
	colorlinks=true,       	% false: ссылки в рамках; true: цветные ссылки
	linkcolor=blue,          % внутренние ссылки
	citecolor=blue,        % на библиографию
	filecolor=magenta,      % на файлы
	urlcolor=cyan           % на URL
}
%----------------------------------------
%Работа с библиографией (как бич)
%----------------------------------------
\renewcommand{\refname}{Список литературы}%Изменение названия списка литературы для article
%\renewcommand{\bibname}{Список литературы}%Изменение названия списка литературы для book и report
%----------------------------------------
\begin{document}
	\section{Кислоты и основания по Аррениусу. Ион гидроксония. Сильные и слабые кислоты и основания. Константы кислотности и основности. Ступенчатая диссоциация на примере фосфорной кислоты.}
	В самом общем определении:
	
	\textbf{Кислота -- вещество которое в реакции отдает ион $H^+$ }
	
	
	\textbf{Основание -- вещество которое в реакции отдает ион $OH^-$}
	
	Это верно для любых растворов и газов. В более узком смысле понимают, что кислота это вещество которое диссоциирует с образованием иона $H^+$ или иона гидроксония $H_3 O^+$ (он очень редкий)
	\begin{equation*}
	\ce{HA + H_2 O <-> A^- + H_3 O^- }
	\end{equation*}
	
	Аналогично для оснований. Только заменить ион водорода на $\ce{OH^-}$
	
	Чтобы понять слабая кислота или основание или нет надо смотреть на константы диссоциации. Напомню, что если есть реакция диссоциации
	\begin{align*}
	A_x B_y = x A + y B
	\end{align*} 
	То константой диссоциации называют
	\begin{align*}
	K = \dfrac{x[A] \cdot y[B]}{[A_x B_y]}
	\end{align*}
	Где в скобках концентрации.
	
	Если $K > 1$ то кислоту или  основание условно считают сильной. И наоборот.
	
	Применительно к кислотам и основаниям константу диссоциации называют константой кислотности и основности соответственно. 
	
	Можно слегка переписать определение. Для разложения 
	\begin{align*}
	\ce{HA <-> H^+ + A^-}
	\end{align*}
	Имеем
	\begin{align}
	K_a = \dfrac{[H^+]^2}{C - [H^+]}
	\end{align}
	Тут мы воспользовались тем, что ионов кислотного остатка и водорода ровно одинаковое количество. И $C$ это сумма ионов вообще. Понятно что она остается неизменной. Величину $pK_a = -\log_{10} K_a$ Называют кислотностью. Для оснований все аналогично, только ионы водорода поменяйте на $OH^-$ Конечно же все эти константы зависят от температуры. Как именно надо смотреть в таблицу.
	
	Диссоциация для многоосновных кислот протекает ступенчато. Например для фосфорной кислоты
	\begin{align*}
	&\ce{H_3PO_4 <-> 	H^+ + H_2PO_4^-} \quad K_1= \dfrac{[H^+][H_2PO_4^{-}]}{[H_3 PO_4]} = 7.25 \cdot 10^{-3} \\
	&\ce{H_2PO_4^- <-> 	H^+ + HPO_4^{2-}} \quad K_2= \dfrac{[H^+][HPO_4^{2-}]}{[H_2 PO_4^{-}]} = 6.31 \cdot 10^{-8}\\
	&\ce{HPO_4^{2-} <-> 	H^+ + PO_4^{3-}} \quad K_3= \dfrac{[H^+][PO_4^{3-}]}{[H PO_4^{2-}]} = 3.98 \cdot 10^{-13}
	\end{align*}
	
	Как видно с каждой ступенью константа на пять порядков меньше. Это более-менее общее свойство. Такое эмпирическое правило называется правилом  Полинга. Произведение всех констант $K_1 K_2 K_3$ соответствует полной диссоциации.  
\end{document}