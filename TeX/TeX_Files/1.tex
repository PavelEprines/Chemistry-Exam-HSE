\section{Вещество. Классификация химических веществ. Химические элементы. Атом, атомный номер, относительная атомная масса, изотопы.}

\textbf{Вещество} --- нечто, состоящее из атомов; нечто, в чем выделение атомов невозможно или теряет физический смысл (например, плазма или звёздное вещество), к предмету рассмотрения химией не относят.

Согласно химической классификации, химические вещества делятся на индивидуальные (чистые) вещества и смеси.

В свою очередь, чистые вещества делятся на простые (состоящие из атомов одного элемента) и сложные. Простые в свою же очередь делятся по своим свойствам на металлы (имеют характерный блеск, высокую теплопроводность и электропроводность, обычно твердые и т.п.) и неметаллы (не обладают блеском, ковкостью, обычно газ и т.п.). Сложные же делятся на органические (соединения углерода) и неорганические (все остальные).

Смеси же делятся на растворы (однородная система с потенциально непостоянным составом, это их отличие от химического соединения) и механические смеси (неоднородные, взвеси).

\textbf{Атом} --- частица вещества микроскопических размеров и массы, наименьшая часть химического элемента, являющаяся носителем его свойств.

\textbf{Атомный номер} (зарядовое число) --- количество протонов в атомном ядре. Равно заряду ядра в единицах элементарного заряда и является порядковым номером соответствующего химического элемента в таблице Менделеева.

Атомная масса измеряется в \textbf{относительных атомных единицах} (а. е. м./углеродная единица) --- единицах массы, определяемых как 1/12 массы свободного покоящегося атома углерода \ce{^12 C}.

\begin{equation}
	1 \text{ а. е. м} = 1.660 539 066 60(5) \cdot 10^{-27} \text{ кг}
\end{equation}

\textbf{Изотопы} --- разновидность атомов одного и того же химического вещества (т.е. атомы имеют одинаковое число протонов), сходные по свойствам (а именно по структуре электронных оболочек), но отличающиеся массой ядер. Изотопы также иногда называются нуклидами. Различие по массе происходит из-за различного числа нейтронов в ядрах.

Масса, которая указывается в таблице Менделеева, является средневзвешенной по распространенности в природе массой изотопов вещества.


