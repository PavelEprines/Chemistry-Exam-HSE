\section{Аллотропные и полиморфные модификации. Основные классы неорганических соединений: оксиды, кислоты, основания, соли, бинарные соединения.}
\subsection{Аллотропные и полиморфные модификации.}
\textbf{Аллотропия} -- существование двух и более простых веществ одного и того же химического элемента. Эти вещества называют \textbf{аллотропическими модификациями} или \textbf{формами}.

Аллотропные модификации присущи веществам в разных агрегатных состояниях, говоря о твёрдых (кристаллических) структурах, довольно часто употребляют синонимичный термин \textbf{полиморфные модификации}.

Самые яркими примерами аллотропии являются модификации таких неметаллов, как углерод, сера или фосфор. 
\begin{itemize}
    \item Для углерода известно множество аллотропных форм, известнейшими являются графит и алмаз. Показательно то насколько различны физические свойства этих модификаций. Графит -- чёрный, матовый крошащийся материал, алмаз -- один из твердейших материалов известных человечеству, при этом отличается блеском и прозрачностью.
    
    Одним из важнейших направлений нанотехнологий сейчас является разработка таких модификаций углерода, как графен, углеродные нанотрубки и фуллерены (последние две формы представлены в очень обширном многообразии, поэтому говорить о точном количестве аллотропных модификаций углерода по сути бессмысленно).
    \item Сера существует в трёх основных аллотропных модификациях -- ромбической, моноклинной (типичные кристаллы) и пластической (спиральные цепочки из атомов серы).
    \item Фосфор так же может существовать в самых различных аллотропных модификациях, самые распространённые из них -- белый, красный и чёрный.
\end{itemize}
Аллотропия характерна не только для ярких неметаллов, она наблюдается и у таких элементнов как бор, кремний и мышьяк, а так же у многих металлов. Один из ярчайших примеров аллотропии металлов -- олово. Олово существует в трех аллотропных модификациях. Серое олово ($\alpha$-Sn) мелкокристаллический порошок, полупроводник, имеющий алмазоподобную кристаллическую решётку, существует при температуре ниже $13,2 {}^{\circ}\mathrm{C}$. Белое олово ($\beta$-Sn) -- пластичный серебристый металл, устойчивый в интервале температур $13,2 - 161 {}^{\circ}\mathrm{C}$. Высокотемпературное гамма-олово ($\gamma$-Sn), имеющее ромбическую структуру, отличается высокой плотностью и хрупкостью, устойчиво между 161 и 232 ${}^{\circ}\mathrm{C}$ (температура плавления чистого олова). 

Соприкосновение серого олова и белого приводит к «заражению» последнего, то есть к ускорению фазового перехода по сравнению со спонтанным процессом из-за появления зародышей новой кристаллической фазы. Совокупность этих явлений называется «оловянной чумой». «Оловянная чума» — одна из причин гибели экспедиции Скотта к Южному полюсу в 1912 году. Путешественники  остались без горючего из-за того, что топливо просочилось из запаянных оловом баков, поражённых «оловянной чумой».

Как говорилось выше аллотропия свойственна не только твёрдым веществам, например газообразный молекулярный кислород на Земле встречается в двух формах \ce{O2} (самая распространённая в атмосфере) и \ce{O3} (озон). Кислород \ce{O2} бесцветен, не имеет запаха; озон имеет выраженный запах, имеет бледно-фиолетовый цвет, он более бактерициден.
\subsection{Основные классы неорганических соединени.}
\textbf{Бинарные соединения} — химические вещества, образованные двумя химическими элементами. Многоэлементные вещества, в формульной единице которых одна из составляющих содержит несвязанные между собой атомы нескольких элементов, а также одноэлементные или многоэлементные группы атомов (кроме гидроксидов и солей), рассматривают как бинарные соединения.

Бинарные соединения — это собирательная группа веществ, которые имеют различное химическое строение. Поэтому их номенклатура может варьироваться в зависимости от генетической принадлежности.

Названия простых бинарных веществ, как правило, образуются добавлением к названию более электроотрицательного элемента суффикса -ид. При необходимости к названиям элементов добавляют кратные приставки или указывают в скобках степень окисления электроположительного элемента без пробела:
\begin{itemize}
\item \ce{SiC} — карбид кремния
\item \ce{KBr} — бромид калия
\item \ce{Fe2O3} — оксид железа(III)
\item \ce{CS2} — дисульфид углерода или сульфид углерода(IV).
\end{itemize}
В сложных бинарных соединениях суффикс -ид добавляется к названиям элементов, находящихся в низших степенях окисления:
\begin{itemize}
\item \ce{PCl3O} — оксид-трихлорид фосфора, или оксихлорид фосфора, трихлороксид фосфора(V)
\item \ce{CrO2Cl2} — диоксид-дихлорид хрома
\item \ce{CCl2O} — оксид-дихлорид углерода или хлорангидрид угольной кислоты, больше известный как фосген.
\end{itemize}
Многие широко известные бинарные соединения носят тривиальные названия, среди них уже приведённый выше фосген, вода \ce{H2O}, аммиак \ce{NH3}, веселящий газ \ce{N2O}, угарный газ \ce{CO}, углекислый газ \ce{CO2}  и другие
Помимо бинарных соединений выделяют следующие классы неорганических соединений:
\begin{itemize}
    \item \textbf{Оксиды} --  бинарное соединение химического элемента с кислородом в степени окисления $-2$, в котором сам кислород связан только с менее электроотрицательным элементом. Химический элемент кислород по электроотрицательности второй после фтора, поэтому к оксидам относятся почти все соединения химических элементов с кислородом. К исключениям относятся, например, дифторид кислорода \ce{OF2}.
    \item \textbf{Кислоты}. Традиционно, при работе с неорганическими кислотами, используют подход Брёнстеда, имея в виду химические соединения, способные отдавать катион водорода. Более обобщённо кислоты -- это соединения, способные принимать электронную пару с образованием ковалентной связи (подход Льюиса). Типичные примеры неорганических кислот: соляная кислота \ce{\color{red}H\color{blue}Cl}, серная кислота \ce{\color{red}H2\color{blue}SO4}, сероводород \ce{\color{red}H2\color{blue}S} или азотная $\color{red}\mathrm{H}\color{blue}\mathrm{NO}_{3}$. Катионы водорода здесь выделены красным цветом, а оставшаяся (синяя) часть называется кислотным остатком. По количеству $n$ потенциально отдаваемых катионов (кислых атомов) водорода кислоты классифицируются, как $n$-основные (например серная кислота является двухосновной). 
    \item \textbf{Основания}. Здесь по аналогии с кислотами есть два подхода. Основание -- химическое соединение, способное образовывать ковалентную связь с протоном (подход Брёнстеда), более обобщённо соединение с вакантной орбиталью другого химического соединения (основание Льюиса). В узком смысле под основаниями понимают основные гидроксиды — сложные вещества, при диссоциации которых в водных растворах отщепляется только один вид анионов — гидроксид-ионы \ce{OH^{-}}. Примерами оснований могут послужить гидроксид цинка \ce{\color{red}Zn\color{blue}(OH)2} или гидроксид калия \ce{\color{red}Ca\color{blue}(OH)2}. Частным случаем оснований являются щёлочи -- основания образованные щелочными или щелочно-земельными металлами и гидроксид группой \ce{OH^{-}}, например гидроксид натрия \ce{\color{red}Na\color{blue}OH} или гидроксид калия \ce{\color{red}K\color{blue}OH}.
    \item \textbf{Соли} -- химические соединения, состоящие из катионов металлов и анионов кислотных остатков. Соли так же могут рассматриваться более обобщённо, в качестве катиона в них могут выступать и другие соединения, например аммоний \ce{NH4^{+}}.
\end{itemize}
Реакции между основаниями и кислотами называются \textbf{реакциями нейтрализации}. Результом такой реакции чаще всего является соль с соответствующим кислотным остатком и основным катионом и слабодиссоциирующе соединение (вода). Например~(\ref{eq:NaOH+HCl->NCl}).
\begin{equation}
\label{eq:NaOH+HCl->NCl}
    \ce{\color{red}Na\color{blue}OH  + \color{red}H\color{blue}Cl -> \color{red}Na\color{blue}Cl}  + \ce{H2O}
\end{equation}
