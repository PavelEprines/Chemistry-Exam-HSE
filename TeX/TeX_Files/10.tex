%
%Не забыть:
%--------------------------------------
%Вставить колонтитулы, поменять название на титульнике



%--------------------------------------

\documentclass[a4paper, 12pt]{article} 

%--------------------------------------
%Russian-specific packages
%--------------------------------------
%\usepackage[warn]{mathtext}
\usepackage[T2A]{fontenc}
\usepackage[utf8]{inputenc}
\usepackage[english,russian]{babel}
\usepackage[intlimits]{amsmath}
\usepackage{esint}
%--------------------------------------
%Hyphenation rules
%--------------------------------------
\usepackage{hyphenat}
\hyphenation{ма-те-ма-ти-ка вос-ста-нав-ли-вать}
%--------------------------------------
%Packages
%--------------------------------------
\usepackage{amsmath}
\usepackage{amssymb}
\usepackage{amsfonts}
\usepackage{amsthm}
\usepackage{latexsym}
\usepackage{mathtools}
\usepackage{etoolbox}%Булевые операторы
\usepackage{extsizes}%Выставление произвольного шрифта в \documentclass
\usepackage{geometry}%Разметка листа
\usepackage{indentfirst}
\usepackage{wrapfig}%Создание обтекаемых текстом объектов
\usepackage{fancyhdr}%Создание колонтитулов
\usepackage{setspace}%Настройка интерлиньяжа
\usepackage{lastpage}%Вывод номера последней страницы в документе, \lastpage
\usepackage{soul}%Изменение параметров начертания
\usepackage{hyperref}%Две строчки с настройкой гиперссылок внутри получаеммого
\usepackage[usenames,dvipsnames,svgnames,table,rgb]{xcolor}% pdf-документа
\usepackage{multicol}%Позволяет писать текст в несколько колонок
\usepackage{cite}%Работа с библиографией
\usepackage{subfigure}% Человеческая вставка нескольких картинок
\usepackage{tikz}%Рисование рисунков
\usepackage{float}% Возможность ставить H в положениях картинки
% Для картинок Моти
\usepackage{misccorr}
\usepackage{lscape}
\usepackage{cmap}

% Для Х И М И И

\usepackage{mhchem}



\usepackage{graphicx,xcolor}
\graphicspath{{Pictures/}}
\DeclareGraphicsExtensions{.pdf,.png,.jpg}

%----------------------------------------
%Список окружений
%----------------------------------------
\newenvironment {theor}[2]
{\smallskip \par \textbf{#1.} \textit{#2}  \par $\blacktriangleleft$}
{\flushright{$\blacktriangleright$} \medskip \par} %лемма/теорема с доказательством
\newenvironment {proofn}
{\par $\blacktriangleleft$}
{$\blacktriangleright$ \par} %доказательство
%----------------------------------------
%Список команд
%----------------------------------------
\newcommand{\grad}
{\mathop{\mathrm{grad}}\nolimits\,} %градиент

\newcommand{\diver}
{\mathop{\mathrm{div}}\nolimits\,} %дивергенция

\newcommand{\rot}
{\ensuremath{\mathrm{rot}}\,}

\newcommand{\Def}[1]
{\underline{\textbf{#1}}} %определение

\newcommand{\RN}[1]
{\MakeUppercase{\romannumeral #1}} %римские цифры

\newcommand {\theornp}[2]
{\textbf{#1.} \textit{ #2} \par} %Написание леммы/теоремы без доказательства

\newcommand{\qrq}
{\ensuremath{\quad \Rightarrow \quad}} %Человеческий знак следствия

\newcommand{\qlrq}
{\ensuremath{\quad \Leftrightarrow \quad}} %Человеческий знак равносильности

\renewcommand{\phi}{\varphi} %Нормальный знак фи

\newcommand{\me}
{\ensuremath{\mathbb{E}}}

\newcommand{\md}
{\ensuremath{\mathbb{D}}}



%\renewcommand{\vec}{\overline}




%----------------------------------------
%Разметка листа
%----------------------------------------
\geometry{top = 3cm}
\geometry{bottom = 2cm}
\geometry{left = 1.5cm}
\geometry{right = 1.5cm}
%----------------------------------------
%Колонтитулы
%----------------------------------------
\pagestyle{fancy}%Создание колонтитулов
\fancyhead{}
%\fancyfoot{}
%----------------------------------------
%Интерлиньяж (расстояния между строчками)
%----------------------------------------
%\onehalfspacing -- интерлиньяж 1.5
%\doublespacing -- интерлиньяж 2
%----------------------------------------
%Настройка гиперссылок
%----------------------------------------
\hypersetup{				% Гиперссылки
	unicode=true,           % русские буквы в раздела PDF
	pdftitle={Заголовок},   % Заголовок
	pdfauthor={Автор},      % Автор
	pdfsubject={Тема},      % Тема
	pdfcreator={Создатель}, % Создатель
	pdfproducer={Производитель}, % Производитель
	pdfkeywords={keyword1} {key2} {key3}, % Ключевые слова
	colorlinks=true,       	% false: ссылки в рамках; true: цветные ссылки
	linkcolor=blue,          % внутренние ссылки
	citecolor=blue,        % на библиографию
	filecolor=magenta,      % на файлы
	urlcolor=cyan           % на URL
}
%----------------------------------------
%Работа с библиографией (как бич)
%----------------------------------------
\renewcommand{\refname}{Список литературы}%Изменение названия списка литературы для article
%\renewcommand{\bibname}{Список литературы}%Изменение названия списка литературы для book и report
%----------------------------------------
\begin{document}
	\section{Обратимые реакции. Химическое равновесие – определение и общие свойства. Константа равновесия. Принцип Ле Шателье. }
	Как известно, химические реакции могут идти только если $\Delta G < 0 $ где $G$ - энергия Гиббса. Если $\Delta G = 0$ то получается реакция может идти в обе стороны. Это возможно если смогут подобраться такие условия, чтобы $\Delta H = T \Delta S$. Учитывая что обе переменные как-то зависят от концентраций дело не безнадежное.
	
	Например 
	\begin{equation*}
	\ce{I_2 + H_2 <-> 2HI}
	\end{equation*}
	
	Пусть есть реакция в равновесии 
	\begin{align*}
	aA + bB + \dots = cC + dD + \dots
	\end{align*}
	Где большие буквы это какие-то формулы, а маленькие это стехиометрические коэффициенты. Тогда константой равновесия называется
	\begin{align*}
	K := \dfrac{[C]^c [D]^d \dots}{[A]^a [B]^b \dots}
	\end{align*}
	Буква в квадратных скобках означает концентрацию. Если это реакция газов, то можно вместо концентраций писать парциальные давления. Они все равно пропорциональны между собой.
	Константа естественно зависит как-то от температуры. От давления тоже, но если оно не очень большое, то обычно это не учитывают.
	
	Она определена именно так, потому что связана с изменением стандартной энергии Гиббса (так называется изменение энергии Гиббса за 1 моль образовавшихся продуктов)
	\begin{align*}
	&\Delta G^0 = - RT \log K\\
	&\Delta G^0_{298} \; (\text{Дж}) = - 5.71 \log K_{298} 
	\end{align*}
	Нижний индекс это температура. Обычно в таблицах пишут константу равновесия именно при комнатных температурах.
	
	Из этой формулы, вспоминая, что $\Delta G = \Delta H - T \Delta S$ получаем
	\begin{align}
	K = \exp \dfrac{T\Delta S^0 - \Delta H }{RT}
	\end{align}
	Это явная зависимость константы равновесия от температуры.
	
	Принцип Ле Шателье, если не пускаться в статфизику гласит тривиальную вещь.
	\textit{Если систему в равновесии подвергнуть внешнему воздействию, то равновесие сместится так, чтобы компенсировать воздействие	}
	Глобально это просто следствие того, что система сидит в потенциальной яме.
	Например, если добавить какого реагента, то реакция усилится в направлении переработки этого реагента.
\end{document}