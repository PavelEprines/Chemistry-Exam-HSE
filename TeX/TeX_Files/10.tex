\begin{document}
	\section{Обратимые реакции. Химическое равновесие – определение и общие свойства. Константа равновесия. Принцип Ле Шателье. }
	Как известно, химические реакции могут идти только если $\Delta G < 0 $ где $G$ - энергия Гиббса. Если $\Delta G = 0$ то получается реакция может идти в обе стороны. Это возможно если смогут подобраться такие условия, чтобы $\Delta H = T \Delta S$. Учитывая что обе переменные как-то зависят от концентраций дело не безнадежное.
	
	Например 
	\begin{equation*}
	\ce{I_2 + H_2 <-> 2HI}
	\end{equation*}
	
	Пусть есть реакция в равновесии 
	\begin{align*}
	aA + bB + \dots = cC + dD + \dots
	\end{align*}
	Где большие буквы это какие-то формулы, а маленькие это стехиометрические коэффициенты. Тогда константой равновесия называется
	\begin{align*}
	K := \dfrac{[C]^c [D]^d \dots}{[A]^a [B]^b \dots}
	\end{align*}
	Буква в квадратных скобках означает концентрацию. Если это реакция газов, то можно вместо концентраций писать парциальные давления. Они все равно пропорциональны между собой.
	Константа естественно зависит как-то от температуры. От давления тоже, но если оно не очень большое, то обычно это не учитывают.
	
	Она определена именно так, потому что связана с изменением стандартной энергии Гиббса (так называется изменение энергии Гиббса за 1 моль образовавшихся продуктов)
	\begin{align*}
	&\Delta G^0 = - RT \log K\\
	&\Delta G^0_{298} \; (\text{Дж}) = - 5.71 \log K_{298} 
	\end{align*}
	Нижний индекс это температура. Обычно в таблицах пишут константу равновесия именно при комнатных температурах.
	
	Из этой формулы, вспоминая, что $\Delta G = \Delta H - T \Delta S$ получаем
	\begin{align}
	K = \exp \dfrac{T\Delta S^0 - \Delta H }{RT}
	\end{align}
	Это явная зависимость константы равновесия от температуры.
	
	Принцип Ле Шателье, если не пускаться в статфизику гласит тривиальную вещь.
	\textit{Если систему в равновесии подвергнуть внешнему воздействию, то равновесие сместится так, чтобы компенсировать воздействие	}
	Глобально это просто следствие того, что система сидит в потенциальной яме.
	Например, если добавить какого реагента, то реакция усилится в направлении переработки этого реагента.
\end{document}