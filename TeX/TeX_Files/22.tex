\section{Галогены. Галогеноводороды. Взаимодействие галогенов с водой. Кислородные соединения галогенов.}

\subsection{Галогены} -- элементы главной подгруппы $VII$ группы: фтор F, хлор Cl, бром Br, иод I и астат At. Встречаются в природе (кроме астата), но только в виде солей из-за высокой химической активности.

Электронная конфигурация галогенов -- $ns^2np^5$: до инртного газа  им недостает одного электрона, поэтому самая характерная их степень окисления -1. 

\paragraph{Простые вещества}

Все галогены состоят из двухатомных молекул, имеют резкий запах и очень ядовиты. В форме простых веществ плохо растворимы в воде (кроме \ce{F2}, который с ней реагирует).

Фтор активно взаимодейтвует с водой, образуя атомарный кислород, который превращается в \ce{O2}:

\begin{equation}
    \ce{F2 + H2O -> 2HF^ +}[\ce{O}]    
\end{equation}

\subsection{Галогеноводороды} -- соединения с полярной коввалентной связью, полярность которой уменьшается в ряду \ce{HF - HCl - HBr - HI}. При обычных условиях все галогеноводороды -- газы, очень хорошо растворимы в воде.

\paragraph{Получение}

\begin{equation}
\ce{CaF2 + H2SO4}\text{(конц.) = }\ce{2HF^ + CaSO4}
\end{equation}

\begin{equation}
\ce{NaCl + H2SO4}\text{(конц.) = } \ce{2HCl^ + NaHSO4}
\end{equation}


\ce{HBr} и \ce{HI} не удается получить таким образом, т.к. они окисляются серной кислотой до чистых вечеств \ce{Br2} и \ce{I2}:

\begin{equation}
\ce{2NaBr + 3H2SO4}\text{(конц.) = }\ce{Br2 + 2NaHSO4 + SO2^ + 2H2O}
\end{equation}

\begin{equation}
\ce{8NaI + 9H2SO4}\text{(конц.) = }\ce{4I2 + 8NaHSO4 + H2S^ + 4H2O}
\end{equation}

Поэтому для их получения используют гидролиз:

\begin{equation}
    \ce{2P + 3Br2 + 6H2O -> 6HBr^ + 2H3PO3}
\end{equation}

\begin{equation}
    \ce{2P + 3I2 + 6H2O -> 6HI^ + 2H3PO3}
\end{equation}

Растворы галогенов в воде это кислоты. Сила кислот увеличивается в ряду \ce{HF - HCl - HBr - HI}. Кроме \ce{HF} все кислоты сильные.

\subsection{Кислородные соединения галогенов}

Все галогены кроме фтора проявляют положительные степени окисления в соединении с кислородом. Фтор во всех кислородных соединениях проявляет степень окисления -1. 

Самые характерные кислородные соединения галогенов - оксиды, кислоты состава $HHalO_n$ (n = 1,2,3,4) и их соли. Наибольшее значение среди этих соединений имеет калиевая соль клорноватой кислоты \ce{KClO3} -- бертолева соль.

Ее получают электролизом:

\begin{equation}
    \ce{KCl + 3H2O ->T[\text{t}] KClO3 + 3H2^}
\end{equation}

При нагревании она разлагается:

\begin{equation}
\ce{4KClO3} = \ce{KCl + 3KClO4} \text{(без катализатора)}
\end{equation}

\begin{equation}
\ce{2KClO3} = \ce{2KCl + 2O2^}\text{(в присутствии \ce{MnO2}) }   
\end{equation}

При нагревании с бертолевой солью многие вещества сгорают в выделяющеся кислороде:

\begin{equation}
    \ce{6P + 5KClO3 -> 3P2O5 + 6 KCl}
\end{equation}

Другие кислородосодержащие соединения галогенов -- также сильные окислители(особенно в кислых средах).

Как правило, они восстанавливаются до галогенид-ионов:

\begin{equation}
\ce{ClO^{-}_n + 2nH^{+} + 2ne- -> Cl- +nH2O}
\end{equation}