\section{Растворы, их классификация. Способы выражения состава раствора – мольная и массовая доли, молярная концентрация. Полярные и неполярные растворители. Отличие свойств растворов от свойств индивидуальных веществ.}

\subsection{Растворы и их классификация}

\textbf{Раствор} --- гомогенные (однородные) смеси переменного состава из двух и более веществ (одно из них --- растворитель). Чаще всего растворы жидкие.

Под переменным составом раствора понимают то простое обстоятельство, что соотношение смешанных друг с другом веществ может непрерывно изменяться в определенных пределах. Например, раствор соли можно разбавлять чистой водой или, наоборот, упаривать, но полученные при этом жидкости в любом случае будут называться растворами соли.

Важная ремарка про \textit{растворитель}: из двух или нескольких компонентов раствора растворителем является тот, который взят в большем количестве и имеет то же агрегатное состояние, что и раствор в целом.

Разумеется, растворять что-то в чем-то можно до определенного предела, а именно до становления раствора \textbf{насыщенным}. Его концентрация будет называться растворимостью при данных температуре и давлении.

Растворы делятся на \textbf{истинные}, \textbf{коллоидные}, а также \textbf{суспензии и эмульсии}.

\textbf{Истинным} растворам фактически дано определение выше.

\textbf{Коллоидные растворы} --- системы, линейные размеры частиц в которой лежат в переделах от 1 до 100 нм.

\textbf{Суспензии} --- система, в которой твердое вещество строго говоря нерастворимо в жидкости (например, глина, взболтанная в воде). Со временем частички выпадут на дно сосуда. Чем меньше частички, тем дольше существует суспензия.

\textbf{Эмульсия} --- система, в которой существуют две жидкости, которые взаимно не смешиваются (например, если налить масла в воду и хорошо взболтать, то получится эмульсия).

\textit{Последние две системы (суспензия и эмульсия) являются двухфазными системами.}

\subsection{Способы выражения состава раствора}

\textbf{Мольная доля} --- способ выражения концентрации как отношения количества интересующего вещества к общему количеству всех веществ раствора (смеси), т.е.:

\begin{equation}
	x_A = \frac{\nu_A}{\sum\limits_i \nu_i}
\end{equation}

\textbf{Массовая доля} --- способ выражения концентрации как отношения массы интересующего вещества к общей массе раствора (смеси), т.е.:

\begin{equation}
	\omega_A = \frac{m_A}{\sum\limits_i m_i}
\end{equation}

\textbf{Молярная концентрация} (молярность) --- количество вещества компонента в единице объема смеси, т.е.:

\begin{equation}
	c_A = [A] = \frac{\nu_A}{V}
\end{equation}

Обычно измеряется в моль/л, единицы измерения обозначают как М, например: 1М раствор \ce{HCl} следует понимать как раствор \ce{HCl}, концентрация \ce{HCl} в котором составляет 1 моль/л. В СИ измеряется в моль/м$^3$.

\subsection{Полярные и неполярные растворители}

\textbf{Полярные растворители} --- растворители, состоящие из полярных молекул, т.е. из молекул, которые обладают электрическим дипольным моментом (например, \ce{H2O} --- вода).

Отдельно уточним, какую формулу молекула точно иметь \textbf{не} может, чтобы быть полярной: 

\begin{itemize}
	\item Правильную тетраэдрическую (\ce{CH4} --- метан)
	
	\item Правильную треугольную (\ce{SO3} --- триоксид серы)
	
	\item Линейную (\ce{CO2} --- углекислый газ)
	
	\item Правильную октаэдрическую (\ce{SF6} --- фторид серы \RN{6})
\end{itemize}

Помимо формы, важна также различная электроотрицательность составляющих молекулу атомов (чтобы молекула приобрела полярность).

Логично, что в полярных растворителях лучше растворяются полярные вещества, в неполярных, соответственно, неполярные.

\subsection{Отличия свойств растворов от свойств индивидуальных веществ}

Свойства растворенных веществ отличатся от свойств чистого вещества в силу появления дополнительных сил межмолекулярных взаимодействий веществ в растворах (например, сил Ван дер Ваальса, водородных связей и т.п.).

